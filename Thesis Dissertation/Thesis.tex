% Set up the document
\documentclass[a4paper, 11pt, oneside]{Thesis}  % Use the "Thesis" style, based on the ECS Thesis style by Steve Gunn
\graphicspath{Figures/}  % Location of the graphics files (set up for graphics to be in PDF format)


% Include any extra LaTeX packages required
%\usepackage[square, numbers, comma, sort&compress]{natbib}  % Use the "Natbib" style for the references in the Bibliography
\usepackage{verbatim}  % Needed for the "comment" environment to make LaTeX comments
\usepackage{vector}  % Allows "\bvec{}" and "\buvec{}" for "blackboard" style bold vectors in maths
\usepackage[backend=biber]{biblatex}
\addbibresource{Bibliography.bib}
\hypersetup{urlcolor=blue, colorlinks=true}  % Colours hyperlinks in blue, but this can be distracting if there are many links.
%% -------------------------------------------------------------
\begin{document}
\frontmatter      % Begin Roman style (i, ii, iii, iv...) page numbering

% Set up the Title Page
\title  {External Kink Modes in Shaped Tokamak Plasmas}
\authors  {\texorpdfstring
            {\href{your web site or email address}{Patrick James Byrne}}
            {Patrick James Byrne}
            }
\addresses  {\groupname\\\deptname\\\univname}  % Do not change this here, instead these must be set in the "Thesis.cls" file, please look through it instead
\date       {\today}
\subject    {}
\keywords   {}

\maketitle
%% ----------------------------------------------------------------

\setstretch{1.3}  % It is better to have smaller font and larger line spacing than the other way round

% Define the page headers using the FancyHdr package and set up for one-sided printing
\fancyhead{}  % Clears all page headers and footers
\rhead{\thepage}  % Sets the right side header to show the page number
\lhead{}  % Clears the left side page header

\pagestyle{fancy}  % Finally, use the "fancy" page style to implement the FancyHdr headers

%% ----------------------------------------------------------------
% Declaration Page required for the Thesis, your institution may give you a different text to place here
\Declaration{

\addtocontents{toc}{\vspace{1em}}  % Add a gap in the Contents, for aesthetics

I, Patrick Byrne, declare that this thesis titled, `External Kink Modes in Shaped Tokamak Plasmas' and the work presented in it are my own. I confirm that:

\begin{itemize} 
\item[\tiny{$\blacksquare$}] This work was done wholly or mainly while in candidature for a research degree at this University.
 
\item[\tiny{$\blacksquare$}] Where any part of this thesis has previously been submitted for a degree or any other qualification at this University or any other institution, this has been clearly stated.
 
\item[\tiny{$\blacksquare$}] Where I have consulted the published work of others, this is always clearly attributed.
 
\item[\tiny{$\blacksquare$}] Where I have quoted from the work of others, the source is always given. With the exception of such quotations, this thesis is entirely my own work.
 
\item[\tiny{$\blacksquare$}] I have acknowledged all main sources of help.
 
\item[\tiny{$\blacksquare$}] Where the thesis is based on work done by myself jointly with others, I have made clear exactly what was done by others and what I have contributed myself.
\\
\end{itemize}
 
 
Signed:\\
\rule[1em]{25em}{0.5pt}  % This prints a line for the signature
 
Date:\\
\rule[1em]{25em}{0.5pt}  % This prints a line to write the date
}
\clearpage  % Declaration ended, now start a new page

%% ----------------------------------------------------------------
% The "Funny Quote Page"
\pagestyle{empty}  % No headers or footers for the following pages

\null\vfill
% Now comes the "Funny Quote", written in italics
There is a theory which states that if ever anyone discovers exactly what the Universe is for and why it is here, it will instantly disappear and be replaced by something even more bizarre and inexplicable. \\
\\
There is another theory which states that this has already happened.

\begin{flushright}
Douglas Adams, \textit{The Restaurant at the End of the Universe}
\end{flushright}

\vfill\vfill\vfill\vfill\vfill\vfill\null
\clearpage  % Funny Quote page ended, start a new page
%% ----------------------------------------------------------------

% The Abstract Page
\addtotoc{Abstract}  % Add the "Abstract" page entry to the Contents
\abstract{
\addtocontents{toc}{\vspace{1em}}  % Add a gap in the Contents, for aesthetics

This thesis represents the first study of MHD external kink modes in shaped plasmas using the high resolution High Beta Tokamak - Extended Pulse (HBT-EP) magnetic sensor set.  This work required construction of a high current, low impedance poloidal field coil, as well as high-current, low-voltage capacitive power supply, and current monitoring hardware and software.  This coil allows shaping of the plasma on the inboard side of the torus, above the midplane, to the point of transitioning the plasma from impingement on a material surface, to being limited by a magnetic x-point.  The construction and operation of the bank is described, with a full partslist appearing in an appendix.  Individual modes in plasmas are extracted from the entire signal using a technique known variously as singular value decomposition, principal component analysis, or biorthogonal decomposition.  Coherent fluctuations of up to three modes is observed using the BD without the need for an a priori choice of basis set.  Measurements of plasma modes are compared to the predictions of simulations involving the codes TokaMac, DCON, and VALEN.  Modes with toroidal wavenumbers n=1 and n=2 are observed across both circular and shaped plasmas.  As predicted by the codes we see a roughly similar poloidal mode spectrum in modes with the same toroidal number, with a pronounced change localized at the x-point.  HBT-EP's active control coil set was used to excite naturally occurring modes, and the response of each type of plasma is compared.  Further studies have looked into excitation of highly stable, low-n, low-m kink modes.  We see a response highly localized at the x-point in shaped plasmas, which has implications for transport control in advanced fusion tokamak configurations.
}

\clearpage  % Abstract ended, start a new page
%% ----------------------------------------------------------------

\setstretch{1.3}  % Reset the line-spacing to 1.3 for body text (if it has changed)

% The Acknowledgements page, for thanking everyone
\acknowledgements{
\addtocontents{toc}{\vspace{1em}}  % Add a gap in the Contents, for aesthetics

The acknowledgements and the people to thank go here, don't forget to include your project advisor\ldots

}
\clearpage  % End of the Acknowledgements
%% ----------------------------------------------------------------

\pagestyle{fancy}  %The page style headers have been "empty" all this time, now use the "fancy" headers as defined before to bring them back


%% ----------------------------------------------------------------
\lhead{\emph{Contents}}  % Set the left side page header to "Contents"
\tableofcontents  % Write out the Table of Contents

%% ----------------------------------------------------------------
\lhead{\emph{List of Figures}}  % Set the left side page header to "List if Figures"
\listoffigures  % Write out the List of Figures

%% ----------------------------------------------------------------
\lhead{\emph{List of Tables}}  % Set the left side page header to "List of Tables"
\listoftables  % Write out the List of Tables
%% ----------------------------------------------------------------
\setstretch{1.5}  % Set the line spacing to 1.5, this makes the following tables easier to read
\clearpage  % Start a new page
\lhead{\emph{Abbreviations}}  % Set the left side page header to "Abbreviations"
\listofsymbols{ll}  % Include a list of Abbreviations (a table of two columns)
{
% \textbf{Acronym} & \textbf{W}hat (it) \textbf{S}tands \textbf{F}or \\
\textbf{LAH} & \textbf{L}ist \textbf{A}bbreviations \textbf{H}ere \\

}

%% ----------------------------------------------------------------
\clearpage  % Start a new page
\lhead{\emph{Physical Constants}}  % Set the left side page header to "Physical Constants"
\listofconstants{lrcl}  % Include a list of Physical Constants (a four column table)
{
% Constant Name & Symbol & = & Constant Value (with units) \\
Speed of Light & $c$ & $=$ & $2.997\ 924\ 58\times10^{8}\ \mbox{ms}^{-\mbox{s}}$ (exact)\\

}

%% ----------------------------------------------------------------
\clearpage  %Start a new page
\lhead{\emph{Symbols}}  % Set the left side page header to "Symbols"
\listofnomenclature{lll}  % Include a list of Symbols (a three column table)
{
% symbol & name & unit \\
$a$ & distance & m \\
$P$ & power & W (Js$^{-1}$) \\
& & \\ % Gap to separate the Roman symbols from the Greek
$\omega$ & angular frequency & rads$^{-1}$ \\
}
%% ----------------------------------------------------------------
% End of the pre-able, contents and lists of things
% Begin the Dedication page

\setstretch{1.3}  % Return the line spacing back to 1.3

\pagestyle{empty}  % Page style needs to be empty for this page
\dedicatory{For/Dedicated to/To my\ldots}

\addtocontents{toc}{\vspace{2em}}  % Add a gap in the Contents, for aesthetics


%% ----------------------------------------------------------------
\mainmatter	  % Begin normal, numeric (1,2,3...) page numbering
\pagestyle{fancy}  % Return the page headers back to the "fancy" style

% Include the chapters of the thesis, as separate files
% Just uncomment the lines as you write the chapters
\lhead{\emph{Introduction}}  % Change the left side page header to "Introduction"
\chapter{Introduction}
\indent The world faces a series of stark choices in the coming years.  The warming of the earth continues to rise, and the rise to accelerate.  As of this writing it is 68$^{\circ}$F and it is also mid-December.  Though there are a series of entrenched interests that resist acknowledging the facts as such, the use of fossil fuels forcing the change is settled science.  However, human development is inextricably linked to the exploitation of energy.  The environmental disruption caused by China's rapid development in the last quarter century is merely an amplified echo of that experienced by the western world during the industrial revolution.  There remain billions of people on earth living in places that have yet to develop, and though raising them to the standard of life enjoyed by the first world would multiply the current consumption of fossil fuels by several times we have no moral right to deny them the lifestyle we enjoy.\par
Given that the use of fossil fuel is already exponentially increasing, and that they are created at a rate many orders of magnitude slower than we are exploiting them, one way or another, at some point in the future, it is guaranteed that humanity will no longer be using fossil fuels to power itself, though that replacement may be wood, peat, or dung, if a more suitable replacement is not found.\par
Carbon-Neutral power generation is thus a must if we are to progress as a species.  The localised, seasonalised and intermittent generation from most renewables disqualifies them until and unless power storage and public grids can catch up.  Hydroelectric is environmentally damaging, and as demonstrated by California's recent drought, subject to the extreme weather caused by global warming.  Nuclear accidents have cost far less in terms of human morbidity than the chronic effects due to burning coal, but the poor public understanding of the physics and technology involved, and the spectacularity with which the failures occur, has made it extremely controversial.\par
Fusion energy is the path out of the bind we find ourselves in with a better future at the end.  The fusion products do not contribute to the greenhouse effect.  The generation is not dependent on seasons or weather.  There is no possibility for a runaway meltdown reaction, and as a new technology with a high degree of enthusiasm among the public, it has none of the perception issues attached to fission nuclear.  One component of the fuel, the hydrogen isotope deuterium, is abundant, with reserves to last several thousands of years.  It is widely distributed, reducing the pressures that lead to conflict over resources.    Though the containment vessel will become activated and radioactive, much like current fission reactors, the reaction products will not be nuclear waster.  Further, there exist more advanced fuel mixes that are completely aneutronic, removing even that slight drawback.  Harnessing fusion is not just a scientific and technical challenge, it is a moral imperative.

\section{The D-T Fusion Reaction}
Fusion occurs if two ions collide with enough energy to overcome the coulomb repulsion of their nuclei until they can come close enough for the strong force interaction to take over, merging the two nuclei.  Different reactions between different reactants and occur at different temperatures, and have larger or smaller cross sections.  The larger the cross section, the smaller the confinement time, or the lower the density required for a certain number of reactions to occur.  Given that the ions in a plasma have a distriubtion of temperatures, the figure of merit for energy generation in a fusion plasma is the value of the so called 'triple product':\begin{equation}
nT\tau_E \geq 5*10^{21} \frac{keV\cdot s}{m^3}
\end{equation} with n being the particle density of the plasma, $\tau_E$ being the energy confinement time, and T being the plasma temperature.  The minimum value in the inequality is for the reaction between deuterium and tritium, which, as the least technically demanding fusion reaction, is the main focus of fusion research.  The values of relevance to tokamak fusion are $T \simeq 10keV$, $n \simeq 10^{20}m^{-3}$, $\tau_E \simeq  5s$ This reaction creates a neutron and a helium-4, or alpha, ion:\begin{equation}
D+T \rightarrow \alpha(3.5MeV) + n(14.3 MeV)
\end{equation}

The magnetically confined helium 'ash' heats the plasma as it thermalizes, and the unconfined neutron carries its heat out of the plasma.  The neutrons are captured in a 'blanket' that absorbs the heat and transfers it to a coolant to generate power.  The blanket can also be given a lithium coating, which on absorption of a neutron generates tritium, helium, and additional energy.\par
\section{Plasma Confinement}
The temperatures required for fusion are above $10^6 K$.  This is far higher than the melting temperature of any available material.  There is therefore a need for a strong temperature gradient between the fusion plasma and any chamber that will contain it.  Diffusion of the plasma, and thus heat transport can be slowed significantly by imposition of a magnetic field in the plasma.  This field will exert a Lorentz force 
\begin{equation}
\vec{F} = q(\vec{E} + \vec{v} \times \vec{B}) 
\end{equation}
on particles moving perpendicular to the direction of the magnetic field.  Viewed along a magnetic field line, charged particles will trace circles of smaller and smaller radii as the field strength is increased.  This radius, known as the Larmor radius, is defined as:\begin{equation}
r_L = \frac{mv_\perp}{qB}
\end{equation}
with m the particle's mass, $v_\perp$ the component of the velocity perpendicular to the magnetic field, q the particle's charge, and B the magnetic field strength.
There is, however, no force felt by a particle traveling along a field line, and so confinement of a plasma in three dimensions is not achievable with straight, uniform magnetic fields.  In tokamaks, an electromagnet generates circular magnetic fields, which is referred to as the toroidal magnetic field.  Plasma is thus free to travel in a periodic fashion along the circular field lines, with diffusion of plasma in a radial direction strongly suppressed.  Plasma diffusion in a vertical direction is also suppressed, but there will be a drift of bulk plasma upwards.  This is due to a net drift of the plasma either up or downwards, caused by the radial gradient in the toroidal field strength.  To counteract this drift, it is necessary to impose a second magnetic field on the plasma.\par 
In tokamaks, this is primarily achieved by passing a large current through the plasma.  This current will add a magnetic field, referred to as the poloidal field, that encircles the plasma.  The resultant field traces a helix through space, and as a plasma element follows this field line the drift will be canceled on average.\par
\subsection{Efficiency and $\beta$}
Generating the toroidal and poloidal fields requires an input of energy, which in a fuison power plant would be deducted against the energy generated.  Given that the rate of energy generation is related to density and pressure, as in Eq \eqref{1}, a useful figure of merit for the efficiency of a fusion power plant is $\beta$, or the ratio of confined plasma pressure to the pressure of the confining magnetic field.  $\beta$ can be expressed with respect to either the toroidal or poloidal magnetic field:
\begin{eqnarray}	
\beta_t = \frac{\langle p \rangle}{\langle B_t \rangle^2/2\mu_0}\\
\beta_p = \frac{\langle p \rangle}{\overline{B}_p^2/2\mu_0}
\end{eqnarray}
Where quantities in brackets $\langle x \rangle$ are the volume averaged, p is the plasma pressure, $B_t$ is the toroidal magnetic field, $\overline{B}_p$ defined as $\frac{\mu_0 I_P}{2 \pi a \kappa}$. The elongation $\kappa$ defined as $\frac{A}{\pi a^2}$ with A the cross-sectional area of the plasma. Further, $\beta_t$ can be normalized against the plasma current, $I_p$, the minor radius $a$, \begin{equation}
\beta_N = \frac{\beta_t a B_t}{I_p}
\end{equation} % Introduction
\lhead{\emph{HBT-EP Capabilities}}  % Change the left side page header to "HBT-EP Capabilities"
\chapter{HBT-EP Capabilities}
\paragraph{}The HBT-EP Tokamak combines a highly configurable configuration with an extremely high resolution magnetic sensor set.  A close-fitting shell, consisting of 20 segments that cover 360$^{\circ}$ of toroidal angle and roughly 180$^{\circ}$ of poloidal angle (from bottom to top, along the outboard side) can be inserted or retracted to change the passive stabilization of the plasma.  On each of the shell segments are mounted two poloidally offset triplets of saddle coils which can be used to apply magnetic fields to the plasma surface for mode excitation or suppression.  Of each triplet, only one coil can be energized at a time, meaning that there are four toroidally complete rings of 10 coils each, in vessel, located less than 2cm from the nominal plasma surface.  Every coil in the system can be energized independently, with up to $\pm$ 40A of current.  This translates to \textbf{HOW MUCH} G at the nominal plasma surface from each coil.  \textbf{Should I include a Niko lot of field on surface?}  Detection of mode activity and response to feedback is accomplished by a set of 216 in-vessel sensors.  These sensors measure both radial and poloidal fields, and are arrayed in two complete poloidal rings, offset by 180 $^{\circ}$ of poloidal angle, one complete toroidal array on the inboard midplane, and four poloidally offset toroidal arrays on the outboard side.  Measurements provided by these sensors will be the basis for all analysis in this paper.
\section{The Plasma}
\subsection{Equlibrium Fields}
\subsection{Material Limiters}
\subsection{Measurement of Equilibrium Parameters}
\section{Plasma Control}
\subsection{Passive Shells}
\subsection{Active Control Coils}
\section{Mode Detection}
\subsection{The Poloidal Arrays}
\subsection{The Toroidal Array}
\subsection{The Feedback Arrays}
\section{Signal Digitization} % HBT-EP Capabilites
\lhead{\emph{The Shaping Coil}}  % Change the left side page header to "The Shaping Coil"
\chapter{The Shaping Coil}
\section{The Coil}
\paragraph{} The shaping coil is a single continuous piece of 1/0 welding cable, run eight times toroidally between the inner ring of the toroidal field magnet cases and the vacuum vessel.  This ex-vessel coil is located slightly above the midplane, mainly due to space limitations.  Were complete freedom of placement to obtain, the coil would likely be placed nearer the midplane, as we have active (pre-programmed) control of the plasma's radial location, but vertical location control is only passively provided by eddies in the conducting vessel walls and shells.
\section{The Coil}
\subsection{The Coil Holders}
\subsection{Leads \& Bundle Connections} 
The construction of the coil out of a single conductor was accomplished through directing the conductor from one bundle to another once the number of required turns had been wound.  This necessitated a poloidal run of wire to connect the two.  At the end of the final bundle, the out-lead cable had to travel poloidally once again to meet the in-lead before becoming a twisted pair.  These non-toroidal runs of current were canceled to the best of our ability by running the return lead back at the same location as the poloidal leads connecting each bundle.  Though the thickness of the cable and the tight spaces involved precluded winding these contrary wires, they were connected with zip ties to ensure as good a canceling as possible.  See Figure 1 for a schematic of the windings and location of the leads.\\
\begin{figure}
\includegraphics[width = \textwidth]{./figures/Coil_winding_schematic.png}\begin{flushleft}
\caption{Rough schematic of the inter-bundle connections and coil leads, as seen by an observer between the vacuum vessel's inboard side and the coil.  The toroidal direction is left/right, poloidal is up/down.}
\end{flushleft}
\label{raw_sig}
\end{figure}
The twisted pair from the power supply rises from the laboratory basement, splits with one lead breaking into the lowest turn, and the other breaking out of the highest turn.  The top lead is therefore unshielded for roughly 8 inches.  There are two points at which the cable is bent back upon itself to begin a new bundle and reverse the direction of the current.  This occurs at the same toroidal location, which is itself the same location as the leads.  The current in the leads travels in the opposite direction from the current in the jumps between bundles.  The lead is thus lashed to the the interbundle connections to both reduce the error fields, and prevent the jXB forces on the cables from causing damage to the coil insulation.
\section{The Power Supply}
\paragraph{}The shaping coil is energized by a pre-programmed two-stage passive capacitive supply.  A 7.5mF, 900V (max) bank provides the startup current, with a time to peak current of $800\mu s$.  As this bank discharges, a high current diode passively switches a second 'crowbar' bank into the circuit.  This bank has a capacitance of 0.6F, and a maximum voltage of 250V.  This much larger bank is capable of providing current enough to sustain a 'flat top' current pulse of $\sim$8kA for $\sim$5ms.  By selecting different voltages on each bank pre-discharge, a variety of different current profiles can be developed.  The 'soft start' of the crowbar bank due to the used of a diode allows for a very smooth transition between the two power banks.  This is compared to the current trace of the vertical field coil in Figure .  The power supplies for all other equilibrium field coils are capacitive in nature, but are switched in by ignitrons, which require a voltage drop between the bank and the line into which they switch the bank.  The rapid inrush current as a capacitor is switched into a line of a different voltage leads to a sharp change in the current, leading to strong eddies induced in the surrounding conducting structures.  The banks are easily charged to their full voltage in less than one minute, much less time than the other equilibirum field coil banks.  There are a variety of safety measures used in the bank's construction.  A discussion, full partslist and circuit schematic can be found in Appendix 
\section{The Limiter}


\section{The Material Limiter}
\paragraph{}The poloidal array sensors are protected by $\frac{1}{16}$'' stainless steel shimstock to protect against damage to the wires or ablation of the plastic forms by plasma impinging.  Under normal operating conditions the last closed flux surface of the plasma is 5mm radially inward from the surface of the sensors' shielding.  The plasma is in general vertically centered on the midplane, moves mostly radially, and disrupts inward, so sensors at the midplane are further protected by the inboard and outboard limiters, which establish the 5mm separation between plasma and sensors, and have much larger thermal mass than the shims that cover the sensors.
\paragraph{}However, this is not the case for sensors higher up that are near the location of the X-point.  These sensors which lack the extra protection of a limiter will see the steady state power flux increase as plasma will be preferentially exhausted along a cone expanding radially outward from the X-point.  Additionally, during disruptions, there will be an upward component to the forces on the plasma.  To ensure that this additional heat load does not cause harm to the sensors, a new set of limiters was machined and installed.  These 'blade' limiters are attached with threaded rods to the flanges of the chamber pieces, which themselves act as limiters on the inboard side.  These threaded rods were spot welded to the flanges during an up to air.  These limiters extend radially inward 5mm from the inner most edge of the x-point localized PA sensors.  The poloidal extent of the sensor is such that shading from the plasma is now provided from the inboard midplane to $\Theta = 90^{\circ}$.  The limiter is made of 3/8" 316 stainless steel
%talk about the limiters when I talk about HBT-EP.  Take the same limiter pic, and add my limiter for ease of reading. % Shaping Coil
\lhead{\emph{Signal Analysis}}  % Change the left side page header to "Signal Analysis"
\chapter{Signal Analysis}
\paragraph{}  The large number of sensors and the rapid sampling of each sensor provides an enormous number of individual measurements of magnetic fields during the plasma's lifetime. The method of Biorthogonal Decomposition is used to isolate coherent fluctuations from this large dataset and provides an orthogonal basis set of modes, that are described in terms of their spatial shape and temporal evolution.  This chapter will describe the method of Biorthogonal Decomposition, how it is applied in this research, its strengths and limitations.

\section{The Biorthogonal Decomposition Algorithm}
\paragraph{}
The biorthogonal decomposition is used on non-square matrices, to produce a basis set of pseudo-eigenvalues and pseudo-eigenvectors.  A given matrix S is decomposed into 
\begin{equation}
S^{\intercal} S = U \Sigma V^{\intercal}
\end{equation}
Where the columns of U and the rows of V are orthogonal bases of the structures, however defined, that make up the rows and columns of the data. In this research, the rows are time series of individual sensor signals, with the signal across all sensors at any instant constituting the columns.  
$$
\begin{pmatrix}
\uparrow& &\uparrow\\
s_1&\cdots&s_N\\
\downarrow & &\downarrow
\end{pmatrix}
 =
\begin{pmatrix}
\uparrow& &\uparrow\\
u_1&\cdots&u_N\\
\downarrow & &\downarrow
\end{pmatrix}
\begin{pmatrix}
\sigma_1& &\\
&\ddots&\\
& &\sigma_N
\end{pmatrix}
\begin{pmatrix}
\leftarrow& v_1&\rightarrow\\
&\vdots&\\
\leftarrow& v_M &\rightarrow
\end{pmatrix}
$$



  can be derived by first multiplying the vector by its transpose.  The resulting matrix is square and is factorable into two independent sets of eigenvectors, which are related to the arrangement of column and row vectors.  
In the case of HBT-EP, the time-domain signal for each sensor is stitched together to give a matrix that has dimensions up to 216 X 5000.  The set of eigenvalues will be equal in size to the smaller dimension.  \par
The set of eigenvectors will be determined arbitrarily, with no a priori assumptions as to the form of the basis functions.  The only constraint is that they must all be orthogonal in both space and time.  That the poloidal modes do not form a clean Fourier basis set is therefore of no concern.  The toroidal and temporal structure, for which sinusoids are a good description, will allow for orthogonal bases which map well onto individual modes regardless of poloidal structure.  Thus modes with non-singular poloidal spectra can be well modeled, which would not be the case if a Fourier basis was imposed.\par
A set of synthetic data is generated and analyzed with the biorthogonal decomposition to show the utility of the method.  The plasma current is modeled as a flattop located at 93cm, lasting 18ms. A 1kHz, m=3, n=1 mode is modeled as growing as $\sqrt{t}$ from zero amplitude at the plasma breakdown and ending with the plasma's disruption.  Further, a single-pole decaying eddy that begins at plasma breakdown is imposed on the four sensors that are most strongly coupled to the eddies in the shells.  Each eddy has a different initial magnitude and decay rate.  The traces of the signals in representative sensors, and contours of the signal across the sensor array is plotted below in \ref{fig:Synthetic_signals}, as well as the sum total of all signals. 

\begin{figure}
\includegraphics[width = \textwidth]{./figures/Artificial_Signal.png}
\caption{Top Row, left: Equilibrium signal as seen by one sensor at the outboard midplane.  Right: The 4 eddy currents, plotted to the same scale as the equilibrium signal, offset vertically to aid viewing and indicate poloidal location.  Second Row, left: Contour plot of equilibrium signal across all sensors.  0 degrees of poloidal angle is measured from the outboard midplane.  Plasma is located at 93cm, intensifying signal at the outboard sensors.  Right: Contour plot of eddies.  Third Row, left:  Growing 1kHz mode as measured by the same sensor measuring equilibrium signal.  Right: Equilibrium, mode, and eddy summed together.  Eddy signal is that labeled 'Eddy 2', located just above 0 degrees poloidal angle.}\label{fig:Synthetic_signals}

\end{figure}

Figure \ref{fig:Synthetic_plus_BD} shows the same plots, with the BD eigenmodes that most closely correspond to each  of the separate signals overplotted.  The BD was taken over a window of 3ms starting 14ms after the breakdown of the plasma.  As will be seen later, intelligent choice of time window can influence the goodness of fit of the BD eigenmodes.

\begin{figure}
\includegraphics[width = \textwidth]{./figures/Artificial_Signal_and_BD.png}\begin{flushleft}
\caption{Same plots as Figure 4.1, with BD modes overplotted.  BD lines are bold and green, except in the plot of the 4 eddy traces, where they are bold and the same color and linestyle as the eddy.}\label{fig:Synthetic_plus_BD}
\end{flushleft}
\end{figure}
An example of real data being given this treatment can be seen below.  A real shot, shot number 89200, 
As seen in figure \ref{fig:PA_sensors_top_3_modes} The signal of all sensors in a poloidal array is put through the BD, with the top three modes isolated.  The (exaggerated) amplitude is plotted on a poloidal cross section.  The development of each mode is clear, with some modes growing as time goes on, and others declining in amplitude.  The remaining signal shows some high wavenumber, low amplitude structure, but analysis of such low level signal can be difficult.

\begin{figure}
\includegraphics[width = \textwidth]{./figures/stripey_flucts_joined_grey_89200_mod.png}\begin{flushleft}
\caption{Left: contour plot of poloidal array raw signal, right: 3 dominant modes, and remainder of the full signal.  Mode structure is clearly visible, as is temporal evolution of mode amplitude.  Remainder shows structure - determining significance of low-order modes requires close analysis on a shot-by-shot basis}\label{fig:PA_sensors_top_3_modes}
\end{flushleft}
\end{figure}

\section{Equilibrium Subtraction}
\paragraph{}
Removing the background equilibrium prior to BD analysis for a real shot has been shown to be important for the purposes of isolating MHD modes.  This was previously done by either taking a sliding or 'boxcar' average of the signal, and treating the resulting time-smoothed signal as the equilibrium, or in certain situations, fitting a subset of the datapoints to a polynomial and then treating the polynomial as the equilibrium.  This is crucial for this research as the imposition of the shaping field significantly changes the equilibrium field near the X-point for sensors nearest the X-point.  % Biorthogonal Decomposition

\input{Chapters/Chapter5} % Results

%\input{Chapters/Chapter6} % Results and Discussion

%\input{Chapters/Chapter7} % Conclusion

%% ----------------------------------------------------------------
% Now begin the Appendices, including them as separate files

\addtocontents{toc}{\vspace{2em}} % Add a gap in the Contents, for aesthetics

\appendix % Cue to tell LaTeX that the following 'chapters' are Appendices
\lhead{\emph{Appendix A}}  % Change the left side page header to "Appendix A"
\input{Appendices/AppendixA}	% Appendix Title

%\input{Appendices/AppendixB} % Appendix Title

%\input{Appendices/AppendixC} % Appendix Title

\addtocontents{toc}{\vspace{2em}}  % Add a gap in the Contents, for aesthetics
\backmatter

%% ----------------------------------------------------------------
%\label{Bibliography}
%\lhead{\emph{Bibliography}}  % Change the left side page header to "Bibliography"
%\bibliographystyle{unsrtnat}  % Use the "unsrtnat" BibTeX style for formatting the Bibliography
%\bibliography{Bibliography}  % The references (bibliography) information are stored in the file named "Bibliography.bib"
\printbibliography
\end{document}  % The End
%% ----------------------------------------------------------------
\chapter{The Shaping Coil}
\section{The Coil}
\paragraph{} The shaping coil is a single continuous piece of 1/0 welding cable, run eight times toroidally between the inner ring of the toroidal field magnet cases and the vacuum vessel.  This ex-vessel coil is located slightly above the midplane, mainly due to space limitations.  Were complete freedom of placement to obtain, the coil would likely be placed nearer the midplane, as we have active (pre-programmed) control of the plasma's radial location, but vertical location control is only passively provided by eddies in the conducting vessel walls and shells.
\section{Error Fields} 
The construction of the coil out of a single conductor was accomplished through directing the conductor from one bundle to another once the number of required turns had been wound.  This necessitated a poloidal run of wire to connect the two.  At the end of the final bundle, the out-lead cable had to travel poloidally once again to meet the in-lead before becoming a twisted pair.  These non-toroidal runs of current were canceled to the best of our ability by running the return lead back at the same location as the poloidal leads connecting each bundle.  Though the thickness of the cable and the tight spaces involved precluded winding these contrary wires, they were connected with zip ties to ensure as good a canceling as possible.  See Figure 1 for a schematic of the windings and location of the leads.\\
\begin{figure}
\includegraphics[width = \textwidth]{./figures/Coil_winding_schematic.png}\begin{flushleft}
\caption{Rough schematic of the inter-bundle connections and coil leads, as seen by an observer between the vacuum vessel's inboard side and the coil.  The toroidal direction is left/right, poloidal is up/down.}
\end{flushleft}
\label{raw_sig}
\end{figure}
The twisted pair from the power supply rises from the laboratory basement, splits with one lead breaking into the lowest turn, and the other breaking out of the highest turn.  The top lead is therefore unshielded for roughly 8 inches.  There are two points at which the cable is bent back upon itself to begin a new bundle and reverse the direction of the current.  This occurs at the same toroidal location, which is itself the same location as the leads.  The current in the leads travels in the opposite direction from the current in the jumps between bundles.  The lead is thus lashed to the the interbundle connections to both reduce the error fields, and prevent the jXB forces on the cables from causing damage to the coil insulation.
\section{The Material Limiter}
\paragraph{}The poloidal array sensors are protected by $\frac{1}{16}$'' stainless steel shimstock to protect against damage to the wires or ablation of the plastic forms by plasma impinging.  Under normal operating conditions the last closed flux surface of the plasma is 5mm radially inward from the surface of the sensors' shielding.  The plasma is in general vertically centered on the midplane, moves mostly radially, and disrupts inward, so sensors at the midplane are further protected by the inboard and outboard limiters, which establish the 5mm separation between plasma and sensors, and have much larger thermal mass than the shims that cover the sensors.
\paragraph{}However, this is not the case for sensors higher up that are near the location of the X-point.  These sensors which lack the extra protection of a limiter will see the steady state power flux increase as plasma will be preferentially exhausted along a cone expanding radially outward from the X-point.  Additionally, during disruptions, there will be an upward component to the forces on the plasma.  To ensure that this additional heat load does not cause harm to the sensors, a new set of limiters was machined and installed.  These 'blade' limiters are attached with threaded rods to the flanges of the chamber pieces, which themselves act as limiters on the inboard side.  These threaded rods were spot welded to the flanges during an up to air.  These limiters extend radially inward 5mm from the inner most edge of the x-point localized PA sensors.  The poloidal extent of the sensor is such that shading from the plasma is now provided from the inboard midplane to $\Theta = 90^{\circ}$.  The limiter is made of 3/8" 316 stainless steel
%talk about the limiters when I talk about HBT-EP.  Take the same limiter pic, and add my limiter for ease of reading.
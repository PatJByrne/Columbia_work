\chapter{HBT-EP Capabilities}
\paragraph{}The HBT-EP Tokamak combines a highly configurable configuration with an extremely high resolution magnetic sensor set.  A close-fitting shell, consisting of 20 segments that cover 360$^{\circ}$ of toroidal angle and roughly 180$^{\circ}$ of poloidal angle (from bottom to top, along the outboard side) can be inserted or retracted to change the passive stabilization of the plasma.  On each of the shell segments are mounted two poloidally offset triplets of saddle coils which can be used to apply magnetic fields to the plasma surface for mode excitation or suppression.  Of each triplet, only one coil can be energized at a time, meaning that there are four toroidally complete rings of 10 coils each, in vessel, located less than 2cm from the nominal plasma surface.  Every coil in the system can be energized independently, with up to $\pm$ 40A of current.  This translates to \textbf{HOW MUCH} G at the nominal plasma surface from each coil.  \textbf{Should I include a Niko lot of field on surface?}  Detection of mode activity and response to feedback is accomplished by a set of 216 in-vessel sensors.  These sensors measure both radial and poloidal fields, and are arrayed in two complete poloidal rings, offset by 180 $^{\circ}$ of poloidal angle, one complete toroidal array on the inboard midplane, and four poloidally offset toroidal arrays on the outboard side.  Measurements provided by these sensors will be the basis for all analysis in this paper.
\section{The Plasma}
\subsection{Equlibrium Fields}
\subsection{Material Limiters}
\subsection{Measurement of Equilibrium Parameters}
\section{Plasma Control}
\subsection{Passive Shells}
\subsection{Active Control Coils}
\section{Mode Detection}
\subsection{The Poloidal Arrays}
\subsection{The Toroidal Array}
\subsection{The Feedback Arrays}
\section{Signal Digitization}
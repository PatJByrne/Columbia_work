\chapter{Introduction}
\paragraph{} The world faces a series of stark choices in the coming years.  The warming of the earth continues to rise, and the rise to accelerate.  As of this writing it is 68$^{\circ}$F and it is also mid-December.  Though there are a series of entrenched interests that resist acknowledging the facts as such, the use of fossil fuels forcing the change is settled science.  However, human development is inextricably linked to the exploitation of energy.  The environmental disruption caused by China's rapid development in the last quarter century is merely an amplified echo of that experienced by the western world during the industrial revolution.  There remain billions of people on earth living in places that have yet to develop, and though raising them to the standard of life enjoyed by the first world would multiply the current consumption of fossil fuels by several times we have no moral right to deny them the lifestyle we enjoy.  
\paragraph{}
Given that the use of fossil fuel is already exponentially increasing, and that they are created at a rate many orders of magnitude slower than we are exploiting them, one way or another, at some point in the future, it is guaranteed that humanity will no longer be using fossil fuels to power itself, though that replacement may be wood, peat, or dung, if a more suitable replacement is not found.
\paragraph{}
Carbon-Neutral power generation is thus a must if we are to progress as a species.  The localised, seasonalised and intermittent generation from most renewables disqualifies them until and unless power storage and public grids can catch up.  Hydroelectric is environmentally damaging, and as demonstrated by California's recent drought, subject to the extreme weather caused by global warming.  Nuclear accidents have cost far less in terms of human morbidity than the chronic effects due to burning coal, but the poor public understanding of the physics and technology involved, and the spectacularity with which the failures occur, has made it extremely controversial.
\paragraph{}
Fusion energy is the path out of the bind we find ourselves in with a better future at the end.  The fusion products do not contribute to the greenhouse effect.  The generation is not dependent on seasons or weather.  There is no possibility for a runaway meltdown reaction, and as a new technology with a high degree of enthusiasm among the public, it has none of the perception issues attached to fission nuclear.  One component of the fuel, the hydrogen isotope deuterium, is abundant, with reserves to last several thousands of years.  It is widely distributed, reducing the pressures that lead to conflict over resources.    Though the containment vessel will become activated and radioactive, much like current fission reactors, the reaction products will not be nuclear waster.  Further, there exist more advanced fuel mixes that are completely aneutronic, removing even that slight drawback.  Harnessing fusion is not just a scientific and technical challenge, it is a moral imperative.

\section{The D-T Fusion Reaction}
\paragraph{}
Fusion occurs if two ions collide with enough energy to overcome the coulomb repulsion of their nuclei until they can come close enough for the strong force interaction to take over, merging the two nuclei.  Different reactions between different reactants and occur at different temperatures, and have larger or smaller cross sections.  The larger the cross section, the smaller the confinement time, or the lower the density required for a certain number of reactions to occur.  Given that the ions in a plasma have a distriubtion of temperatures, the figure of merit for energy generation in a fusion plasma is the value of the so called 'triple product':\begin{equation}
nT\tau_E \geq 5*10^{21} \frac{keV\cdot s}{m^3}
\end{equation} with n being the particle density of the plasma, $\tau_E$ being the energy confinement time, and T being the plasma temperature.  The minimum value in the inequality is for the reaction between deuterium and tritium, which, as the least technically demanding fusion reaction, is the main focus of fusion research.  The values of relevance to tokamak fusion are $T \simeq 10keV$, $n \simeq 10^{20}m^{-3}$, $\tau_E \simeq  5s$ This reaction creates a neutron and a helium-4, or alpha, ion:\begin{equation}
D+T \rightarrow \alpha(3.5MeV) + n(14.3 MeV)
\end{equation}

The magnetically confined a helium 'ash' heats the plasma as it thermalizes, and the unconfined neutron carries its heat out of the plasma.  The neutrons are captured in a 'blanket' that absorbs the heat and transfers it to a coolant to generate power or can be captured by a lithium coating, generating tritium, helium, and additional energy.
\paragraph{}
It is not enough, however to merely generate energy.  More energy must be generated than was used to heat and confine the plasma, and the imbalance must be large enough that the excess can be sold at a low enough cost to be competitive, and at high enough volume to underwrite the operation of the plant and provide a profit for the operators.  The most common measurement of tokamak efficiency is $\beta$, which is the ratio of the plasma pressure to magnetic pressure.  $\beta$ can be expressed with respect to either the toroidal or poloidal magnetic field, or the combination:
\begin{eqnarray}	
\beta_t = \frac{\langle p \rangle}{\langle B_t \rangle^2/2\mu_0}\\
\beta_p = \frac{\langle p \rangle}{\overline{B}_p^2/2\mu_0}
\end{eqnarray}
Where $\langle p \rangle$ is the volume averaged plasma pressure, $B_t$ is the toroidal magnetic field, $B_p$ is the poloidal magnetic field,Further, $\beta_t$ can be normalized against the plasma current, $I_p$, the minor radius $a$, \begin{equation}
\beta_N = \frac{\beta_T a B_t}{I_p}
\end{equation}
\section{plasma confinement}

\subsection{A Subsection}

Donec urna leo, vulputate vitae porta eu, vehicula blandit libero. Phasellus eget massa et leo condimentum mollis. Nullam molestie, justo at pellentesque vulputate, sapien velit ornare diam, nec gravida lacus augue non diam. Integer mattis lacus id libero ultrices sit amet mollis neque molestie. Integer ut leo eget mi volutpat congue. Vivamus sodales, turpis id venenatis placerat, tellus purus adipiscing magna, eu aliquam nibh dolor id nibh. Pellentesque habitant morbi tristique senectus et netus et malesuada fames ac turpis egestas. Sed cursus convallis quam nec vehicula. Sed vulputate neque eget odio fringilla ac sodales urna feugiat.

\section{Another Section}

Phasellus nisi quam, volutpat non ullamcorper eget, congue fringilla leo. Cras et erat et nibh placerat commodo id ornare est. Nulla facilisi. Aenean pulvinar scelerisque eros eget interdum. Nunc pulvinar magna ut felis varius in hendrerit dolor accumsan. Nunc pellentesque magna quis magna bibendum non laoreet erat tincidunt. Nulla facilisi.

Duis eget massa sem, gravida interdum ipsum. Nulla nunc nisl, hendrerit sit amet commodo vel, varius id tellus. Lorem ipsum dolor sit amet, consectetur adipiscing elit. Nunc ac dolor est. Suspendisse ultrices tincidunt metus eget accumsan. Nullam facilisis, justo vitae convallis sollicitudin, eros augue malesuada metus, nec sagittis diam nibh ut sapien. Duis blandit lectus vitae lorem aliquam nec euismod nisi volutpat. Vestibulum ornare dictum tortor, at faucibus justo tempor non. Nulla facilisi. Cras non massa nunc, eget euismod purus. Nunc metus ipsum, euismod a consectetur vel, hendrerit nec nunc.
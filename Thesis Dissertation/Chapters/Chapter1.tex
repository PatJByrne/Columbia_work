\chapter{Introduction}
\indent The world faces a series of stark choices in the coming years.  The warming of the earth continues to rise, and the rise to accelerate.  As of this writing it is 68$^{\circ}$F and it is also mid-December.  Though there are a series of entrenched interests that resist acknowledging the facts as such, the use of fossil fuels forcing the change is settled science.  However, human development is inextricably linked to the exploitation of energy.  The environmental disruption caused by China's rapid development in the last quarter century is merely an amplified echo of that experienced by the western world during the industrial revolution.  There remain billions of people on earth living in places that have yet to develop, and though raising them to the standard of life enjoyed by the first world would multiply the current consumption of fossil fuels by several times we have no moral right to deny them the lifestyle we enjoy.\par
Given that the use of fossil fuel is already exponentially increasing, and that they are created at a rate many orders of magnitude slower than we are exploiting them, one way or another, at some point in the future, it is guaranteed that humanity will no longer be using fossil fuels to power itself, though that replacement may be wood, peat, or dung, if a more suitable replacement is not found.\par
Carbon-Neutral power generation is thus a must if we are to progress as a species.  The localised, seasonalised and intermittent generation from most renewables disqualifies them until and unless power storage and public grids can catch up.  Hydroelectric is environmentally damaging, and as demonstrated by California's recent drought, subject to the extreme weather caused by global warming.  Nuclear accidents have cost far less in terms of human morbidity than the chronic effects due to burning coal, but the poor public understanding of the physics and technology involved, and the spectacularity with which the failures occur, has made it extremely controversial.\par
Fusion energy is the path out of the bind we find ourselves in with a better future at the end.  The fusion products do not contribute to the greenhouse effect.  The generation is not dependent on seasons or weather.  There is no possibility for a runaway meltdown reaction, and as a new technology with a high degree of enthusiasm among the public, it has none of the perception issues attached to fission nuclear.  One component of the fuel, the hydrogen isotope deuterium, is abundant, with reserves to last several thousands of years.  It is widely distributed, reducing the pressures that lead to conflict over resources.    Though the containment vessel will become activated and radioactive, much like current fission reactors, the reaction products will not be nuclear waster.  Further, there exist more advanced fuel mixes that are completely aneutronic, removing even that slight drawback.  Harnessing fusion is not just a scientific and technical challenge, it is a moral imperative.

\section{The D-T Fusion Reaction}
Fusion occurs if two ions collide with enough energy to overcome the coulomb repulsion of their nuclei until they can come close enough for the strong force interaction to take over, merging the two nuclei.  Different reactions between different reactants and occur at different temperatures, and have larger or smaller cross sections.  The larger the cross section, the smaller the confinement time, or the lower the density required for a certain number of reactions to occur.  Given that the ions in a plasma have a distriubtion of temperatures, the figure of merit for energy generation in a fusion plasma is the value of the so called 'triple product':\begin{equation}
nT\tau_E \geq 5*10^{21} \frac{keV\cdot s}{m^3}
\end{equation} with n being the particle density of the plasma, $\tau_E$ being the energy confinement time, and T being the plasma temperature.  The minimum value in the inequality is for the reaction between deuterium and tritium, which, as the least technically demanding fusion reaction, is the main focus of fusion research.  The values of relevance to tokamak fusion are $T \simeq 10keV$, $n \simeq 10^{20}m^{-3}$, $\tau_E \simeq  5s$ This reaction creates a neutron and a helium-4, or alpha, ion:\begin{equation}
D+T \rightarrow \alpha(3.5MeV) + n(14.3 MeV)
\end{equation}

The magnetically confined helium 'ash' heats the plasma as it thermalizes, and the unconfined neutron carries its heat out of the plasma.  The neutrons are captured in a 'blanket' that absorbs the heat and transfers it to a coolant to generate power.  The blanket can also be given a lithium coating, which on absorption of a neutron generates tritium, helium, and additional energy.\par
\section{Plasma Confinement}
The temperatures required for fusion are above $10^6 K$.  This is far higher than the melting temperature of any available material.  There is therefore a need for a strong temperature gradient between the fusion plasma and any chamber that will contain it.  Diffusion of the plasma, and thus heat transport can be slowed significantly by imposition of a magnetic field in the plasma.  This field will exert a Lorentz force 
\begin{equation}
\vec{F} = q(\vec{E} + \vec{v} \times \vec{B}) 
\end{equation}
on particles moving perpendicular to the direction of the magnetic field.  Viewed along a magnetic field line, charged particles will trace circles of smaller and smaller radii as the field strength is increased.  This radius, known as the Larmor radius, is defined as:\begin{equation}
r_L = \frac{mv_\perp}{qB}
\end{equation}
with m the particle's mass, $v_\perp$ the component of the velocity perpendicular to the magnetic field, q the particle's charge, and B the magnetic field strength.
There is, however, no force felt by a particle traveling along a field line, and so confinement of a plasma in three dimensions is not achievable with straight, uniform magnetic fields.  In tokamaks, an electromagnet generates circular magnetic fields, which is referred to as the toroidal magnetic field.  Plasma is thus free to travel in a periodic fashion along the circular field lines, with diffusion of plasma in a radial direction strongly suppressed.  Plasma diffusion in a vertical direction is also suppressed, but there will be a drift of bulk plasma upwards.  This is due to a net drift of the plasma either up or downwards, caused by the radial gradient in the toroidal field strength.  To counteract this drift, it is necessary to impose a second magnetic field on the plasma.\par 
In tokamaks, this is primarily achieved by passing a large current through the plasma.  This current will add a magnetic field, referred to as the poloidal field, that encircles the plasma.  The resultant field traces a helix through space, and as a plasma element follows this field line the drift will be canceled on average.\par
\subsection{Efficiency and $\beta$}
Generating the toroidal and poloidal fields requires an input of energy, which in a fuison power plant would be deducted against the energy generated.  Given that the rate of energy generation is related to density and pressure, as in Eq \eqref{1}, a useful figure of merit for the efficiency of a fusion power plant is $\beta$, or the ratio of confined plasma pressure to the pressure of the confining magnetic field.  $\beta$ can be expressed with respect to either the toroidal or poloidal magnetic field:
\begin{eqnarray}	
\beta_t = \frac{\langle p \rangle}{\langle B_t \rangle^2/2\mu_0}\\
\beta_p = \frac{\langle p \rangle}{\overline{B}_p^2/2\mu_0}
\end{eqnarray}
Where quantities in brackets $\langle x \rangle$ are the volume averaged, p is the plasma pressure, $B_t$ is the toroidal magnetic field, $\overline{B}_p$ defined as $\frac{\mu_0 I_P}{2 \pi a \kappa}$. The elongation $\kappa$ defined as $\frac{A}{\pi a^2}$ with A the cross-sectional area of the plasma. Further, $\beta_t$ can be normalized against the plasma current, $I_p$, the minor radius $a$, \begin{equation}
\beta_N = \frac{\beta_t a B_t}{I_p}
\end{equation}